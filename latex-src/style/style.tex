% =========================================================================
% = SNET THESIS TEMPLATE STYLE
% =========================================================================

% http://www.snet.tu-berlin.de
% ------------------------
% Adapted version from http://hci.rwth-aachen.de/karrer_thesistemplate (Thorsten Karrer)
% Further adaptions for http://www.elearn.rwth-aachen.de (Sascha Hoellger)
% Further adaptions for SNET @ TU Berlin by Sebastian Göndör (sebastian.goendoer@tu-berlin.de)


% =========================================================================
% = CHANGELOG
% =========================================================================
% [0.1.9]
% - Fixed styling for chapters and toc using Komascript
% - Remove double bibliography TOC entry
%
% [0.1.8]
% - fixed "warning UFT8 is used". biblatex requires ascii encoding; by Dirk
%
% [0.1.7]
% replaced "Titelsec" commands (and whole package) by appropriate KOMA-Script commands; by Dirk
%
% [0.1.6]
% replaced deprecated \rm commands with \rmfamily commands; by Dirk
%
% [0.1.4b]
% backend=biber added in line 139
%
% [0.1.4a]
% title page: image logo sizes and margins adjusted to printable area
% removed separation of online and offline references
%
% [0.1.3]
% wider text body
% added "school" to the titlepage
% paragraph indents
% correctly placed footnote graphics
%
% [0.1.2]
% new titlepage
% some minor fixes
%
% [0.1.1]
% changed titlepage logo
% added listoffigures and listoftables
% excluded abstract from toc
% no (roman) numbering for frontmatter
%
% [0.1]
% adapted version 0.991b from sascha hoellger @ rwth aachen


% =========================================================================
% = MISC
% =========================================================================

\usepackage{a4wide}					%
\usepackage{verbatim}				%
\usepackage[toc,page]{appendix}			%
\usepackage[withpage]{acronym}			%
\usepackage{amsthm}				% Definitions


% =========================================================================
% = COLORS
% =========================================================================

\usepackage{xcolor}					% Colors
\definecolor{LightBlue}{rgb}{0.55,0.55,1}
\definecolor{DarkBlue}{rgb}{0.2,0.2,0.5}
\definecolor{DarkRed}{rgb}{0.71,0.12,0.07}

% =========================================================================
% = PAGE LAYOUT
% =========================================================================

\usepackage{geometry}
\geometry{inner=3cm, outer=2cm, bottom=4cm}

\newcommand{\setwidesite}				% changes the geometry to have less margin
{
	\fancyhfoffset[LE,RO]{0cm}
	\fancyheadoffset[LO,RE]{0cm}
	\fancyfootoffset[RE]{2cm}
	\newgeometry{inner=2cm, outer=2cm, bottom=4cm}
}

\usepackage{style/noindent}				%do not indent at new paragraphs but add a vertical offset

\setlength{\parindent}{4mm}
\setlength{\parskip}{1.5mm }

%Line spacing
\renewcommand{\baselinestretch}{1.5}

% =========================================================================
% = TYPESETTING
% =========================================================================

\usepackage[hyphens]{url}				% url
\usepackage{hyphenat}				% hyphenation. use \hyphenation{}

\righthyphenmin=5
\lefthyphenmin=5


% =========================================================================
% = TABLE OF CONTENTS
% =========================================================================

\setcounter{secnumdepth}{4}
\setcounter{tocdepth}{3}

\addtokomafont{disposition}{\rmfamily}


% =========================================================================
% = FONTS
% =========================================================================

\usepackage{mathpazo}
\usepackage[scaled=.95]{helvet}
\usepackage{courier}


% =========================================================================
% = SYMBOLS
% =========================================================================

%\usepackage{gensymb}
\usepackage{textcomp} 				% for \textmu (non-italic $\mu$)
\makeatletter						% this makes "@" a regular letter


% =========================================================================
% = TABLES
% =========================================================================

\usepackage{tabularx}
\usepackage{booktabs}
\usepackage{multirow}
\usepackage{longtable}				% tables spanning over more than one page

%%\setlength{\fboxsep}{0mm}			% spacing between \fbox border and content

\usepackage{amsmath}				% math fonts
\usepackage{amssymb}				% math symbols
\usepackage{setspace}				% line spacing


% =========================================================================
% = BIBILOGRAPHY
% =========================================================================

% 2018-10-16 - changed to use Bibtex instead

%\usepackage[style=numeric,natbib=true,backend=biber]{biblatex}

% apparently no effect?
%\renewcommand{\bibsetup}{
%	\markboth{
%		\MakeUppercase{Bibliography}
%	}{}
%}

%\ifdefined\bibheadingonline
%  \defbibheading{online}{\section*{\bibheadingonline}}
%\else
%  \defbibheading{online}{\section*{Online References}}
%\fi
%\ifdefined\bibheadingoffline
%  \defbibheading{offline}{\section*{\bibheadingoffline}}
%\else
%  \defbibheading{offline}{\section*{Printed References}}
%\fi
%
%\defbibfilter{online}{%
%  \( \type{online} \)}
%
%\defbibfilter{offline}{%
%  \( \not \type{online} \)}
%
%\bibliography{Bibliography}


% =========================================================================
% = LANGUAGE & ENCODING
% =========================================================================

\usepackage[english]{babel}				 
%\usepackage[ngerman]{babel}

\selectlanguage{english}
%\selectlanguage{ngerman}

\usepackage[T1]{fontenc}
\usepackage[utf8]{inputenc}				% can use native umlauts

% \usepackage[babel,german=quotes]{csquotes}	% provides \enquote{Blupp} => "`Blupp"'
\usepackage[babel,english=american]{csquotes}	% provides \enquote{Blupp} => "`Blupp"'

%\SetCiteCommand{\parencite}			% Changed for biblatex

\usepackage{units}					% unified way of setting values with units

\usepackage{appendix}


% =========================================================================
% = CODE LISTINGS
% =========================================================================

\usepackage{listings}

% Listings Styles from Max

\definecolor{violet}{cmyk}{0.45,0.97,0.27,0.21}
\definecolor{lstblue}{cmyk}{1,0.80,0,0}
\definecolor{lstgreen}{cmyk}{0.71,0.21,0.65,0.22}
\definecolor{bluegrey}{cmyk}{0.56,0.24,0.11,0.05}
\definecolor{javadoc}{cmyk}{0.88,0.59,0,0}
\definecolor{lstgrey}{cmyk}{0.55,0.44,0.42,0.32}

\lstdefinelanguage{SQL}{
     keywords={},
     keywordstyle=\color{bluegrey}\bfseries,
     morekeywords=[2]{CREATE,TABLE,IF,NOT,EXISTS,NULL,SET,DEFAULT,PRIMARY,KEY,COLLATE,CHARACTER,AUTO_INCREMENT,ENGINE,CHARSET},
     keywordstyle={[2]\color{violet}\bfseries},
     otherkeywords={int,varchar,double,text,tinyint},
     sensitive=false,
     morecomment=[l][\color{lstgreen}]{//},
     morecomment=[s][\color{lstgreen}]{/*}{*/},
     morecomment=[s][\color{javadoc}]{/**}{*/},
     morestring=[b]',
     morestring=[b]"
  }
\lstdefinelanguage{PHP}{
     keywords={},
     keywordstyle=\color{bluegrey}\bfseries,
     morekeywords=[2]{static,function,if,return,pow,sin,cos,asin,min,sqrt,int},
     keywordstyle={[2]\color{violet}\bfseries},
     otherkeywords={@param, @returns, @author, @type, @link, @see},
     sensitive=false,
     morecomment=[l][\color{lstgreen}]{//},
     morecomment=[s][\color{lstgreen}]{/*}{*/},
     morecomment=[s][\color{javadoc}]{/**}{*/},
     morestring=[b]',
     morestring=[b]"
  }
\lstdefinelanguage{JavaScript}{
     keywords={},
     keywordstyle=\color{bluegrey}\bfseries,
     morekeywords=[2]{attributes, class, classend, do, empty, endif, endwhile, fail, function, functionend, if, implements, in, inherit, inout, not, of, operations, out, return, set, then, types, while, use},
     keywordstyle={[2]\color{violet}\bfseries},
     otherkeywords={@param, @returns, @author, @type, @link, @see},
     sensitive=false,
     morecomment=[l][\color{lstgreen}]{//},
     morecomment=[s][\color{lstgreen}]{/*}{*/},
     morecomment=[s][\color{javadoc}]{/**}{*/},
     morestring=[b]',
     morestring=[b]"
  }
\lstdefinelanguage{Java}{
     keywords={},
     keywordstyle=\color{bluegrey}\bfseries,
     morekeywords=[2]{abstract,boolean,break,byte,case,catch,char,class,
      const,continue,default,do,double,else,extends,false,final,
      finally,float,for,goto,if,implements,import,instanceof,int,
      interface,label,long,native,new,null,package,private,protected,
      public,return,short,static,super,switch,synchronized,this,throw,
      throws,transient,true,try,void,volatile,while},
     keywordstyle={[2]\color{violet}\bfseries},
     morekeywords=[3]{@SuppressWarnings, @Capability, @Override},
     keywordstyle={[3]\color{lstgrey}},
     otherkeywords={@param, @return, @returns, @author, @link, @see},
     sensitive,
     morecomment=[l]//,
     morecomment=[s]{/*}{*/},
     morecomment=[s][\color{javadoc}]{/**}{*/},
     morestring=[b]",
     morestring=[b]',
  }[keywords,comments,strings]

% some listings styles from Gregor Aisch
% http://vis4.net/blog/2009/09/noch-mehr-sprach-definitionen-fuer-latex-listings/

\lstdefinelanguage{HTML5} {morekeywords={a, abbr, address, area, article, aside, audio, b, base, bb, bdo, blockquote,  body, br, button, canvas, caption, cite, code, col, colgroup, command, datagrid, datalist, dd, del, details, dialog, dfn, div, dl, dt, em, embed, eventsource, fieldset, figure, footer,  form,  h1, h2,  h3,  h4, h5,  h6,  head,  header,  hr, html,  i, iframe,  img,  input,  ins, kbd,  label,  legend,  li,  link,  mark,  map,  menu,  meta,  meter,  nav,  noscript,  object,  ol,  optgroup,  option,  output,  p,  param,  pre,  progress,  q,  ruby,  rp,  rt,  samp,  script,  section,  select,  small,  source,  span,  strong,  style,  sub,  sup,  table,  tbody,  td,  textarea,  tfoot,  th,  thead,  time,  title,  tr,  ul,  var,  video},
sensitive=false, morecomment=[s]{<!--}{-->}, morestring=[b]", morestring=[d]'}

\lstdefinelanguage{CSS} {morekeywords={azimuth,  background-attachment,  background-color,  background-image,  background-position,  background-repeat,  background,  border-collapse,  border-color,  border-spacing,  border-style,  border-top, border-right, border-bottom, border-left,  border-top-color, border-right-color, border-bottom-color, border-left-color,  border-top-style, border-right-style, border-bottom-style, border-left-style,  border-top-width, border-right-width, border-bottom-width, border-left-width,  border-width,  border,  bottom,  caption-side,  clear,  clip,  color,  content,  counter-increment,  counter-reset,  cue-after,  cue-before,  cue,  cursor,  direction,  display,  elevation,  empty-cells,  float,  font-family,  font-size,  font-style,  font-variant,  font-weight,  font,  height,  left,  letter-spacing,  line-height,  list-style-image,  list-style-position,  list-style-type,  list-style,  margin-right, margin-left,  margin-top, margin-bottom,  margin,  max-height,  max-width,  min-height,  min-width,  orphans,  outline-color,  outline-style,  outline-width,  outline,  overflow,  padding-top, padding-right, padding-bottom, padding-left,  padding,  page-break-after,  page-break-before,  page-break-inside,  pause-after,  pause-before,  pause,  pitch-range,  pitch,  play-during,  position,  quotes,  richness,  right,  speak-header,  speak-numeral,  speak-punctuation,  speak,  speech-rate,  stress,  table-layout,  text-align,  text-decoration,  text-indent,  text-transform,  top,  unicode-bidi,  vertical-align,  visibility,  voice-family,  volume,  white-space,  widows,  width,  word-spacing,  z-index},
sensitive=false, morecomment=[s]{/*}{*/}, morestring=[b]", morestring=[d]'}

\lstdefinelanguage{JavaFX} {morekeywords={abstract, after, and, as, assert, at, attribute, before, bind, bound, break, catch, class, continue, def, delete, else, exclusive, extends, false, finally, first, for, from, function, if, import, indexof, in, init, insert, instanceof, into, inverse, last, lazy, mixin, mod, new, not, null, on, or, override, package, postinit, private, protected, public-init, public, public-read, replace, return, reverse, sizeof, static, step, super, then, this, throw, trigger, true, try, tween, typeof, var, where, while, with },
sensitive=false, morecomment=[l]{//}, morecomment=[s]{/*}{*/}, morestring=[b]", morestring=[d]'}

\lstdefinelanguage{MXML} {morekeywords={mx:Accordion, mx:Box, mx:Canvas, mx:ControlBar, mx:DividedBox, mx:Form, mx:FormHeading, mx:FormItem, mx:Grid, mx:GridItem, mx:GridRow, mx:HBox, mx:HDividedBox, mx:LinkBar, mx:Panel, mx:TabBar, mx:TabNavigator, mx:Tile, mx:TitleWindow, mx:VBox, mx:VDividedBox, mx:ViewStack, mx:Button, mx:CheckBox, mx:ComboBase, mx:ComboBox, mx:DataGrid, mx:DateChooser, mx:DateField, mx:HRule, mx:Image, mx:Label, mx:Link, mx:List, mx:Loader, mx:MediaController, mx:MediaDisplay, mx:MediaPlayback, mx:MenuBar, mx:NumericStepper, mx:ProgressBar, mx:RadioButton, mx:RadioButtonGroup, mx:Spacer, mx:Text, mx:TextArea, mx:TextInput, mx:Tree, mx:VRule, mx:VScrollBar, mx:Application, mx:Repeater, mx:UIComponent, mx:UIObject, mx:View, mx:FlexExtension, mx:UIComponentExtension, mx:UIObjectExtension, mx:Fade, mx:Move, mx:Parallel, mx:Pause, mx:Resize, mx:Sequence, mx:WipeDown, mx:WipeLeft, mx:WipeRight, mx:WipeUp, mx:Zoom, mx:EventDispatcher, mx:LowLevelEvents, mx:UIEventDispatcher, mx:CurrencyFormatter, mx:DateFormatter, mx:NumberFormatter, mx:PhoneFormatter, mx:ZipCodeFormatter, mx:CursorManager, mx:DepthManager, mx:DragManager, mx:FocusManager, mx:HistoryManager, mx:LayoutManager, mx:OverlappedWindows, mx:PopUpManager, mx:SystemManager, mx:TooltipManager, mx:CreditCardValidator, mx:DateValidator, mx:EmailValidator, mx:NumberValidator, mx:PhoneNumberValidator, mx:SocialSecurityValidator, mx:StringValidator, mx:ZipCodeValidator, mx:DownloadProgressBar, mx:ArrayUtil, mx:ClassUtil, mx:Delegate, mx:ObjectCopy, mx:URLUtil, mx:XMLUtil, mx:CSSSetStyle, mx:CSSStyleDeclaration, mx:CSSTextStyles, mx:StyleManager, mx:HTTPService, mx:RemoteObject, mx:Service},
sensitive=false, morecomment=[s]{<!--}{-->}, morestring=[b]", morestring=[d]'}

\lstdefinelanguage{LZX} {morekeywords={a, alert, animator, animatorgroup , attribute, audio , axis, axisstyle , b, barchart, basebutton , basebuttonrepeater , basecombobox , basecomponent , basedatacombobox , basedatepicker , basedatepickerday , basedatepickerweek , basefloatinglist , basefocusview , baseform , baseformitem , basegrid , basegridcolumn , baselist , baselistitem , basescrollarrow , basescrollbar , basescrollthumb , basescrolltrack , baseslider , basestyle , basetab , basetabelement , basetabpane , basetabs , basetabsbar , basetabscontent , basetabslider , basetrackgroup , basetree , basevaluecomponent , basewindow , br , button , canvas , chart , chartbgstyle , chartstyle , checkbox , class , columnchart , combobox , command , connection , connectiondatasource , constantboundslayout , constantlayout , datacolumn , datacombobox , datalabel , datamarker , datapath , datapointer , dataselectionmanager , dataseries , dataset , datasource , datastyle , datastylelist , datatip , datepicker , debug , dragstate , drawview , edittext , event , face , floatinglist , font , font , form , frame , grid , gridcolumn , gridtext , handler , hbox , horizontalaxis , hscrollbar , i , image , img , import , include , inputtext , javarpc , label , labelstyle , layout , legend , library , linechart , linestyle , list , listitem , LzTextFormat , menu , menubar , menuitem , menuseparator , method , modaldialog , multistatebutton , node , p , param , piechart , piechartplotarea , plainfloatinglist , plotstyle , pointstyle , pre , radiobutton , radiogroup , rectangularchart , regionstyle , remotecall , resizelayout , resizestate , resource , reverselayout , richinputtext , rpc , script , scrollbar , security , selectionmanager , sessionrpc , simpleboundslayout , simpleinputtext , simplelayout , slider , soap , splash , stableborderlayout , state , statictext , style , submit , swatchview , SyncTester , tab , tabelement , tabpane , tabs , tabsbar , tabscontent , tabslider , Test , TestCase , TestResult , TestSuite , text , textlistitem , tickstyle , tree , u , valueline , valuelinestyle , valuepoints , valuepointstyle , valueregion , valueregionstyle , vbox , verticalaxis , view , view , vscrollbar , webapprpc , window , windowpanel , wrappinglayout , XMLHttpRequest , xmlrpc , zoomarea},
sensitive=false, morecomment=[s]{<!--}{-->}, morestring=[b]", morestring=[d]'}

\lstset{
  numbers=left,
  numberstyle=\tiny,
  numbersep=5pt,
  breaklines=true,
  stepnumber=1,
  tabsize=2,
  basicstyle=\ttfamily\small,
  frame=none,
  numberfirstline=true,
  firstnumber=1,
  keywordstyle=\color{violet}\bfseries,
  ndkeywordstyle=\color{bluegrey}\bfseries,
  identifierstyle=\color{black},
  commentstyle=\color{lstgreen}\ttfamily,
  stringstyle=\color{lstblue}\ttfamily,
  showstringspaces=false
}


% ========================================================================
% = CHANGE LIST DEFINITIONS
% ========================================================================

% change color of item list
\renewcommand{\labelitemi}{\color{black}$\bullet$}
\renewcommand{\labelitemii}{\color{black}$\circ$}
\renewcommand{\labelitemiii}{\color{black}$\ast$}
\renewcommand{\labelitemiv}{\color{black}$\diamond$}

% change color of enum list
\renewcommand{\labelenumi}{\color{black}\arabic{enumi}.}
\renewcommand{\labelenumii}{\color{black}\alph{enumii})}
\renewcommand{\labelenumiii}{\color{black}\roman{enumiii}.}
\renewcommand{\labelenumiv}{\color{black}\Alph{enumiv}.}

% change color of description list
\usepackage{enumitem}
\setdescription{font=\color{black}\rmfamily\itshape}
% \renewenvironment{description}{\list{font=\color{DarkRed}\itshape}}{\endlist}


% ========================================================================
% = FOOTNOTES
% ========================================================================

% change color of footnotes
\renewcommand{\thefootnote}{\color{black}\arabic{footnote}}

% use nice footnote indentation
\deffootnote[1em]{1em}{1em}{\textsuperscript{\thefootnotemark}\,}


% =========================================================================
% = GRAPHICS AND IMAGES
% =========================================================================

\usepackage{graphicx}
\graphicspath{{images/}}				% path to your image folder

\usepackage{eso-pic}					% needed for the full-face titlepage
\usepackage{chngpage}				% we need this to determine if a figure is on an odd or even page
\usepackage{tikz}					% tikz pictures

% captions of tables and images
\usepackage[hang,small,sf]{caption}
\renewcommand{\captionfont}{\sffamily\small}
\renewcommand{\captionlabelfont}{\bfseries}

\usepackage{float}
\usepackage{placeins}
% \floatstyle{ruled}
%\floatplacement

\renewcommand{\floatpagefraction}{0.85}		% if a figure takes more than 85% of a page it will be typeset on a separate page
\usepackage[it,bf,tight,hang,raggedright]{subfigure}

%\numberwithin{figure}{section}
%\numberwithin{table}{section}


% =========================================================================
% = HEADER
% =========================================================================

\newcommand{\STYLEfootnotetext}
{
  \begin{minipage}
  {.2\textwidth}
   
  \end{minipage}
}

% Change page headers and footers:
\usepackage{calc}
\usepackage{fancyhdr}
\pagestyle{fancy}
\fancyhfoffset[RO,LE]{0.1cm} %{\marginparsep+\marginparwidth}
\fancyhfoffset[RE,LO]{0.1cm}
%\fancyheadoffset[RE,LO]{\hoffset + \oddsidemargin}
\renewcommand{\headrule}{{\color{black}%
  \hrule width\headwidth height\headrulewidth \vskip-\headrulewidth}}
\fancyhf{}
\fancyhead[RE]{\slshape \nouppercase{\leftmark}}    % Even page header: "page   chapter"
\fancyhead[LO]{\slshape \nouppercase{\rightmark}}   % Odd  page header: "section   page"
\fancyhead[RO,LE]{\bfseries \thepage}

%- \fancyfoot[LE]{\STYLEleftpicture}
%- \fancyfoot[RO]{\STYLErightpicture}
\fancyfoot[LE]{\STYLEfootnotetext}

\renewcommand{\headrulewidth}{1pt}    % Underline headers
\renewcommand{\footrulewidth}{0pt}

% =========================================================================
% = SECTIONS THEMING
% =========================================================================

\newcommand{\allsectionformat}{\color{black}\rmfamily\normalfont}

% Font style and colors
\addtokomafont{part}{\Huge\allsectionformat}
\addtokomafont{chapter}{\Huge\allsectionformat}
\addtokomafont{section}{\allsectionformat}
\addtokomafont{subsection}{\allsectionformat}
\addtokomafont{subsubsection}{\allsectionformat}
\addtokomafont{paragraph}{\allsectionformat}
\addtokomafont{subparagraph}{\allsectionformat}

% Spacing before and after the section titles
\RedeclareSectionCommand[
  beforeskip=-.75\baselineskip,
  afterskip=.5\baselineskip]{section}

\RedeclareSectionCommand[
  beforeskip=-5\baselineskip,
  afterskip=.5\baselineskip]{chapter}


% =========================================================================
% = TYPESETTING - TWEAKES
% =========================================================================

\addtokomafont{section}{\LARGE}
\addtokomafont{subsection}{\large}

% instead of sloppy
%\tolerance 1414
%\hbadness 1414
%- \tolerance 2414
%- \hbadness 2414
%- \emergencystretch 1.5em
%- \hfuzz 0.3pt
%- \widowpenalty=10000     % Hurenkinde r
%- \clubpenalty=10000      % Schusterjungen
%- \brokenpenalty=10000
%- \interlinepenalty=9000 % seitenumbruch im absatz
%- \vfuzz \hfuzz
%- \raggedbottom


% =========================================================================
% =  USER DEFINED COMMANDS
% =========================================================================

\newcommand{\chapterquote}[2]{
    \begin{quotation}
    \begin{flushright}
    \noindent\emph{``{#1}''\\[1.5ex]---{#2}}
    \end{flushright}
    \end{quotation}
}
